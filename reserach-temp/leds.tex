\section{Assignment 1}
Show that the ESP32 can turn three different LEDs on and off seperately using an internal loop with delays.\\

\subsection{Circuit}
For this assignment, the circuit used is quite straightforward.
Connect three LEDs to the ESP32, put these LEDs in series with a resistor, and connect them to a common ground.
For all subsequent assignments, it can be assumed that the same circuit is used unless explicitly stated otherwise.

\begin{figure}[htbp]
    \centering
    \begin{circuitikz}
        \ctikzset{multipoles/dipchip/pin spacing=0.2}
        \ctikzset{resistor=european}
        \draw (0,0) node[dipchip,
            num pins=30,
            hide numbers,
            external pins width=0.1,
            external pad fraction=3](C){ESP32};

            \draw (C.pin 11) -- ++(-0.5,0)   to[stroke led] ++(0,1) to[R] ++(0,2.0);
            \draw (C.pin 12) -- ++(-1.5,0) coordinate (P1) -- (P1|-C.pin 11) to[stroke led] ++(0,1.0) to[R] ++(0,2.0);
            \draw (C.pin 13) -- ++(-2.5,0) coordinate (P2) -- (P2|-C.pin 11) to[stroke led] ++(0,1.0) to[R] ++(0,2.0);
            \draw (C.pin 14) -- ++(-3.5,0) coordinate (P3) -- (P3|-C.pin 11) -- ++(0,3.0) -- ++(3.0,0);

            \node [right,font=\tiny]
            at (C.bpin 11) {GPIO27};

            \node [right,font=\tiny]
            at (C.bpin 12) {GPIO14};

            \node [right,font=\tiny]
            at (C.bpin 13) {GPIO12};

            \node [right,font=\tiny]
            at (C.bpin 14) {GND};
    \end{circuitikz}
\end{figure}

\subsection{Toolchain}
For this assignment, I decided on using the ESP-IDF toolchain.
More specifically, I used the \code{idf.py} command line tool in combination with a text editor.
I never used the ESP-IDF toolchain before and was curious about what it would be like.
One thing I really like about the ESP-IDF toolchain is that it provides a \cpp{} compiler that supports both language and library features of more recent \cpp{} standards, which is something the Arduino IDE lacks.

\clearpage

\subsection{Software}

\inputminted{cpp}{../leds/main/leds.cpp}
\clearpage

