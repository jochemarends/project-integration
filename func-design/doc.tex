\documentclass{article}

\usepackage[hidelinks]{hyperref}
\usepackage[outputdir=build]{minted}
\usepackage[toc,acronym,nonumberlist,style=super]{glossaries}
\usepackage{apacite}
%\usepackage[numbers]{natbib} % (IEEE)
\usepackage{circuitikz}
\usepackage{graphicx}
\usepackage{microtype}
\usepackage{texlogos}

\renewcommand{\familydefault}{\sfdefault}

% Acronyms go here.
\newacronym{acs}{ACS}{Applied Computer Science}
\newacronym{aiot}{AIoT}{Artificial Intelligence of Things}
\newacronym{rd}{R\&D}{Research and Development}
\newacronym{http}{HTTP}{Hypertext Transfer Protocol}
\newacronym{api}{API}{Application Programming Interface}
\newacronym{fram}{F-RAM}{Ferroelectric Random-Access Memory}
\newacronym{dram}{DRAM}{Dynamic RAM}
\newacronym{psram}{PSRAM}{Pseudostatic RAM}
\newacronym{sram}{PSRAM}{Static RAM}

% Glossaries go here.
\makeglossaries

\newglossaryentry{wifi}{
    name=WiFi,
    description={A wireless networking technology that uses radio waves to provide high-speed Internet access},
}

\newglossaryentry{bluetooth}{
    name=Bluetooth,
    description={A short-range wireless interconnection of mobile phones, computers, and other electronic devices},
}



\title{Project Integration \\ Functional Design}
\author{Jochem Arends \\ 495637 \\ Group 1}
\date{Academic Year: 2023-2024}

\begin{document}

\maketitle
\newpage

\tableofcontents
\clearpage

\section{Fall Detection}

While classifying a fall might be a trivial task for humans, the concept of a fall is difficult to describe, making it hard to express in the form of a compter algorithm \cite{noury-2007}.
When developing a fall detection system, an important choice that has to be made is what sensor(s) to use.
When sensor(s) do not provide sufficient data, a system may not be able to distinguish between a fall and normal activities.
Several fall detection techniques were identified and their pros and cons are weighed below.

\subsection{Gyroscope and Accelerometer}

Wearable devices commonly use a gyroscope, an acceleration meter, or a combination of both for fall detection \cite{delahoz-2014}.
A gyroscope is a type of sensor that measures angular velocity \cite{passaro-2017}.
By integrating this angular velocity, the orientation of an object can be obtained.
This data can be used by a wearable system to determine whether the user is in a lying position.
An acceleration meter (also known as an accelerometer) is a sensor that measures linear velocity.

\subsection{Barometric Pressure Sensor}

A barometric pressure sensor measures atmospheric pressure.
Because atmospheric pressure increases as altitude decreases, observing this data can be used in fall detection systems \cite{sun-2019}.

\section{Data Storage}

Having concrete ideas about where and how data will be stored is crucial for the design of the product.
In this section, the concern is not only about dynamic data such as user data and sensor readings, but also about firmware and credentials.
Several data storage options were identified and their pros and cons are weighed below.

\subsection{Relational Database}

In a relational database, data is organized into rows and columns, which together form a table \cite{ibm-rd}.
SQL is often used for interacting with a relational database, offering a widely known syntax for querying data.
When working with relational databases, one is often not concerned about the underlying format the database uses for storing the data.
Relational databases are a suitable option for storing dynamic data such as user data and sensor readings.

\subsection{Non-Relational Database}

In a non-relational datbase, data is not organized into rows and columns \cite{microsoft-nrd}.
Instead, data may be stored as JSON, graphs, or other formats \cite{microsoft-nrd}.
A non-relational database 
A relational database can be easier to setup and in some cases data can be stored more efficiently.

\subsection{Non-Volatile Memory}
Non-volatile memory is a type of computer memory that retains stored data when power is disconnected, making it suitable for storing data such as credentials and firmware.
Non volatile memory generally has slower write speeds, and has a reduced write endurance compared to volatile memory \cite{cintra-2013}.

\subsection{Flash}
Flash memory is an electronic non-volatile computer memory storage meidum that can be electrically erased and reprogrammed.

%\subsection{F-RAM}

%\subsection{PS-RAM}

\clearpage

\bibliographystyle{apacite}
\bibliography{research}

\end{document}

