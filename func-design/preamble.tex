\usepackage[hidelinks]{hyperref}
\usepackage[outputdir=build]{minted}
\usepackage[toc,acronym,nonumberlist,style=super]{glossaries}
\usepackage{apacite}
\usepackage[numbers]{natbib} % (IEEE)
%\usepackage{circuitikz}
\usepackage{graphicx}
\usepackage{microtype}
\usepackage{texlogos}
\usepackage{tikz}

\graphicspath{{images/}}

\usetikzlibrary{automata, positioning, arrows}

\newcommand{\cpp}{C\nolinebreak\hspace{-.05em}\raisebox{.4ex}{\tiny\bf +}\nolinebreak\hspace{-.10em}\raisebox{.4ex}{\tiny\bf +}}

\tikzset{
    ->, % makes the edges directed
    >=stealth', % makes the arrow heads bold
    node distance=3cm, % specifies the minimum distance between two nodes. Change if necessary.
    every state/.style={thick, fill=gray!10}, % sets the properties for each ’state’ node
    initial text=$ $, % sets the text that appears on the start arrow
}

\renewcommand{\familydefault}{\sfdefault}

% Acronyms go here.
\newacronym{acs}{ACS}{Applied Computer Science}
\newacronym{aiot}{AIoT}{Artificial Intelligence of Things}
\newacronym{rd}{R\&D}{Research and Development}
\newacronym{http}{HTTP}{Hypertext Transfer Protocol}
\newacronym{api}{API}{Application Programming Interface}
\newacronym{fram}{F-RAM}{Ferroelectric Random-Access Memory}
\newacronym{dram}{DRAM}{Dynamic RAM}
\newacronym{psram}{PSRAM}{Pseudostatic RAM}
\newacronym{sram}{PSRAM}{Static RAM}

% Glossaries go here.
\makeglossaries

\newglossaryentry{wifi}{
    name=WiFi,
    description={A wireless networking technology that uses radio waves to provide high-speed Internet access},
}

\newglossaryentry{bluetooth}{
    name=Bluetooth,
    description={A short-range wireless interconnection of mobile phones, computers, and other electronic devices},
}

