\section{Final Assignment}

For the final assignment, I am managing a database using SQL and the ESP32.
I chose for this assignment because of the likelihood that the ESP32 has to interact with a database for the final product.
The goal of this assignment is that the ESP32 will be able to log certain events, which will be stored into a database.
The only event that will be logged is the pressing of a push button.
The ESP32 will not communicate directly with the database; instead, it communicates with the database via a web server.

\section{Demonstration Setup}

\begin{figure}[htbp]
    \centering
    \begin{circuitikz}
        \ctikzset{multipoles/dipchip/pin spacing=0.2}
        \ctikzset{resistor=european}
        \draw (0,0) node[dipchip,
            num pins=30,
            hide numbers,
            external pins width=0.1,
            external pad fraction=3](C){ESP32};

            \draw (C.pin 15) -- ++(-0.5,0) -- ++(0,-0.5) -- ++(3,0) -- ++(0,5) -- ++(-3,0) coordinate (P1);
            \draw (C.pin 14) -- ++(-0.5,0) to[R] ++(-2.0,0) -- ++(0,1.5) to[push button] ++(0,1) coordinate (P2) -- (P2|-P1) -- (P1);
            \draw (C.pin 11) -- ++(-2.5,0);

            \node [right,font=\tiny]
            at (C.bpin 11) {GPIO27};

            \node [right,font=\tiny]
            at (C.bpin 14) {GND};

            \node [right, font=\tiny]
            at (C.bpin 15) {5V};
    \end{circuitikz}
\end{figure}

\noindent
I used the circuit above to connect the push button to the ESP32.
The locations of the pins may differ depending on which version of the ESP32 is being used.
The web server and the database are hosted a Raspberry Pi.
I configured port forwarding so that the ESP32 can communicate with the web server from other networks.

\subsection{Used Software Libraries}

For the client, I have used libraries that come with ESP-IDF.
These libraries allowed me among other things to establish a \gls{wifi} connection and perform \gls{http} requests.
For the server, I have used Go's standard library packages working with \gls{http} and SQL.
I have also used a package that enables MySQL support for Go's \href{https://pkg.go.dev/database/sql}{database/sql} standard library package.
a Go driver for MySQL, which can be found at: \url{https://github.com/go-sql-driver/mysql}.
I made use of Go's standard library for working with \gls{http}.

\subsection{Software Architecture}

The software consists of three major components: a \gls{http} client, a \gls{http} server, and a relational database.
The ESP32 is communicating with the database via a web server.
The client makes \gls{http} requests to the server, after which server will query the database based on the request being made. Of these components, the  server and the database are hosted a Raspberry Pi, while the  client runs on an ESP32.

The \gls{http} client communicates with the \gls{http} server, which in turn communicates with the database.
The ESP32 is interacting with the database via a web server as hosting a SQL client would be a very intensive task for a microcontroller.

\subsection{Results}

I did not experience any problems when working on this assignment.
However, I did come accross some challenges for the design.
At first instance I thought 
A SQL client is 

\subsection{Accountability}

I have done this assignment by myself.

\clearpage

